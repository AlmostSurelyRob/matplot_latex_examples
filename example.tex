\documentclass{article}

\date{\today} 

\usepackage{amsmath}
\usepackage{amsfonts}
\usepackage{authblk}
\usepackage{booktabs}
\usepackage{fontspec}
\usepackage{graphicx}
\usepackage{hyperref}
\usepackage{multirow}
\usepackage{rotating}
\usepackage{sectsty}
\usepackage{pgf}

%\addbibresource{references.bib}
%Get rid of these nasty lines around links
\hypersetup{pdfborder=0 0 0} 

\author{Robert Sawko}
\affil{Department of Engineering Computing, Cranfield University}
\title{Examples with Python PGF support}

%Setting a nicer(?) font (which need xetex)
\defaultfontfeatures{Scale=MatchLowercase,Mapping=tex-text}
\setmainfont{Liberation Serif}

\begin{document}
\maketitle

\begin{abstract}

  This document presents examples usage of \texttt{matplotlib} together with
  \LaTeX. We will use the PGF in order to create low-level drawing macros which
  will be compiled by \texttt{xelatex}. We will also cover a generation of
  figures and legends separately which in some situations is beneficial. 
\end{abstract}

\section{Introduction}
This document presents the results of PGF output and \LaTeX symbiosis. It also
gives guidelines on how to achieve it. It is divided as follows.

\subsection{Prerequisites}
I tried to keep the document minimal so that you should not need many packages
in order to run the example. These are the essential bits
\begin{enumerate}
  \item \texttt{python2-matplotlib}
\end{enumerate}

\section{Example figures}

Several examples are presented here.

\begin{figure}[b] 
  \centering
  \input{graphics/c40.pgf} 
  \input{graphics/c60.pgf} 
  \input{graphics/c80.pgf} 
  \\
  \input{graphics/legend.pgf} 
  \caption{Several figures and a legend}
  \label{fig:problem} 
\end{figure}

\section{Future work}
\begin{enumerate}
  \item Include an automated build system like Make or CMake.
\end{enumerate}
\end{document}
