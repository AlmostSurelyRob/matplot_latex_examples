\documentclass[a4paper,12pt]{article}

\date{\today}

\usepackage{amsmath}
\usepackage{amsfonts}
\usepackage{authblk}
\usepackage{booktabs}
\usepackage{fontspec}
\usepackage{graphicx}
\usepackage{hyperref}
\usepackage{pgf}

%\addbibresource{references.bib}
%Get rid of these nasty lines around links
\hypersetup{pdfborder=0 0 0}

\author{Robert Sawko}
\affil{Department of Engineering Computing, Cranfield University}
\title{Python and PGF}

%Setting a nicer(?) font (which need xetex)
\defaultfontfeatures{Scale=MatchLowercase,Mapping=tex-text}
\setmainfont{Liberation Serif}

\begin{document}
\maketitle

\begin{abstract}

  This document presents examples usage of \texttt{matplotlib} together with
  \LaTeX. We will use the PGF in order to create low-level drawing macros which
  will be compiled by \texttt{xelatex}. We will also cover a generation of
  figures and legends separately which in some situations is beneficial.
\end{abstract}

\section{Introduction}
This document is a response to one of the questions about \texttt{matplotlib}
capability to generate graphs in \LaTeX\ format. A potential benefit of this is
a better integration of the document with the graphs as the fonts, font sizes
etc.\ will be obtained during the compilation of the document.

The integration is achieved via PGF backend. To find out more about backends
read \href{http://matplotlib.org/faq/usage_faq.html#what-is-a-backend}{this}.
PGF backend is described \href{http://matplotlib.org/users/pgf.html}{here}. PGF
stands for Portable Network Graphics - more can be found here.

Admittedly, \texttt{matplotlib} is a bit more involved than \texttt{gnuplot}.
You often need to invoke more commands. On the other hand it is more flexible
and can be used in any python code with potential loops over directory
structure. This cannot be achieved with \texttt{gnuplot} without some scripting
which quickly gets more involved than a readable Python code. Also,
\texttt{matplotlib} seems to offer more functionality and is more flexible.
Personally, I use \texttt{gnuplot} for interactive plotting as I can obtain
fairly complex plots straight from the files. I would use \texttt{matplotlib}
for larger documents.

Note that I tried to keep the setup simple without a use of fancy build system
or self configuration tools. Graph generation and document compilation could be
run with a single command. The use would become very simple but the
configuration, the scripts etc would have become more complex. For now the
purpose is to demonstrate PGF and \LaTeX integration and show some good
practices. In the future I may add branches for people who would like to see how
to run this with \texttt{make} or \texttt{cmake}.

\section{Example figures}

Several examples are presented here. On figure~\ref{fig:latex} we are simply
looking at the capability of plotting via pgf. Figure~\ref{fig:several} we are
mainly looking at using legend as separate graph that is only tied to the actual
plots at the level of the document. This approach is beneficial if you are
plotting many figures of the same type and the legend would have been repeated
on each graph. Also, having legend outside of the plot makes the plot clearer
and there is nothing to obscure the results.


\begin{figure}[t]
  \centering
  %note that we're using \input!
  \input{graphics/bessel.pgf}
  \\
  \caption{I am printing word "abscissa" just to prove that the fonts are
  matched. We're also admiring a \LaTeX\ equations on the graph.}
  \label{fig:latex}
\end{figure}
\begin{figure}[p]
  \begin{center}
    \input{graphics/ReG0150ReL0150.pgf}
    \input{graphics/ReG1500ReL0150.pgf}
    \\
    \vspace{5mm}
    \input{graphics/ReG0150ReL1500.pgf}
    \input{graphics/ReG1500ReL1500.pgf}
    \\
    \input{graphics/legend.pgf}
  \end{center}
  \caption{Several figures and a legend. Note the convenience of having the
  legend in the actual document. It doesn't have to be assigned to any subplot.}
  \label{fig:several}
\end{figure}

\section{Future work}
\begin{enumerate}
  \item Include an automated build system like Make or CMake.
  \item Include different branches for advanced and novice users.
  \item Write up a proper reference section.
\end{enumerate}
\end{document}
